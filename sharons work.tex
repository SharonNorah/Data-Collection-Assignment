\documentclass[10pt,letterpaper]{article}
\begin{document}
\title{ 
A CASE STUDY OF IDENTITY THEFT IN UGANDA
}
\author{by AKOL SHARON NORAH  \\ 216000979 \\ 16/U/74}
\maketitle
\section{Introduction }
Identity theft has been described as the newest form of theft. Using only a name
and a social security number, an identity thief can borrow money, acquire credit, obtain
employment, or even attain a criminal record (Office of the Inspector General, 1999;
Federal Trade Commission, 2000, 2001). Victims are often unaware of their
victimization. Calls from collection agencies and denied loans are frequently the first
signs of trouble. Victims of identity theft have to cope with the frustration of having their
privacy invaded, their financial well being threatened, and astonishingly few resources to
turn to for assistance. While I work with you to help you protect your identity from theft and fraud,this is no foolproof method to prevent identity theft. Credit monitoring and early detection of new account openings under your name can help you reacrt to attempted fraud and help to minimise damage. If u have become a victim of identity theft, it is important that you have a service that works with you to restore your statement.


\section{Problem statement }
Investing in an identity theft protection services is beneficial but the most important is helping you when you need it if you become an identity victim. Sorting out the mess that results from identity theft can be stressful,time consuming, costly and overwhelming. There is paper work to fill out, phone calls to be made and documents to collect. Luckily if you have an identity theft protection from a reputable company, you probably have a resource to turn to in order to begin the process of restoring your identity.it is important to have a helping hand if you become an identity theft victim that is why am creating a personalised approachto privancy and identity protectionwith comprehensive restoration help.
\section{GOAL AND OBJECTIVES OF THE STUDY GOAL:}
 The purpose of this study is to examine the magnitude and characteristics of identity theft..
And how to investigate and resolve complicated trails of fraudulent activity  and restore a lost identity and ensure safety of private information of an identity theft victim.
SPECIFIC OBJECTIVES OF THE STUDY. 
	To determine if government official’s claims and the media’s portrayal of the substantial rise in identity theft incidents are supported empirically.
	To know the kind of information  required from an identity theft victim inorder to help him or her restore his or her idengtity.
	To know how to provide personalized guidance and action steps when a victim confirms fraudulent activity.
	To determine the costs of following  up an identity theft victim

\section{RESEARCH QUESTIONS }
The following are the basic questions addressed in  the study.
1.	Is the increase in identity theft similar to those of the other theft related offenses?

2.	Is the clearance rate for identity theft similar to those of the other theft related of-fenses?

3.	What are the predominant demographic characteristics associated with victims and offenders of identify theft?

4.	What are the risks that come with trying to restore a lost or stolen identity.


5.	How long does it take to trace for a stolen identity.

\section{SCOPE OF STUDY}
The study seeks to  find out the possibilities of restoring lost identities and securing the iden-tity victim’s private information. The study will focus  urban areas in Uganda.

\section{LIMITATIONS OF STUDY}
 Unfortunately, when using official data such as police records or citizen complaints, a re-searcher must appreciate that the information recorded is not always a true reflection of past events. This concept is covered in criminology under the label the dark figure of crime, which argues that there is a degree of crime that often goes unreported. With identity theft, part of that dark figure comes from the way this offense is viewed by federal and state law enforcement. There is an inherent inaccuracy in cases recorded as identity theft.

 There has only been a brief period of time for these cases to amass within the department’s records.

Personnel in the records department, from which data was collected, have stated that there was a time lag between their new database coming online and the police department’s per-sonnel switching over to it

\section{CONCLUSION }
In view of the above summary, it was evident that identity theft had increased beyond any-ones expectation and there the demand to end it is highly appreciated



\end{document}